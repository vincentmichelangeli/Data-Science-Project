%%%%%%%%%%%%%%%%%%%%%%%%%%%%%%%%%%%%%%%%%%%%%%%%%%%%%%%%%%%%%%%%%%%%%
%% PREAMBLE

\documentclass[a4paper, 12pt]{article}

% General document formatting
\usepackage[margin=1.25in]{geometry}
\usepackage[parfill]{parskip}
\usepackage[utf8]{inputenc}

% Figures
\usepackage{graphicx}
\usepackage[section]{placeins} 
% References 
\usepackage[authoryear,round]{natbib}
    
% Related to math
\usepackage{amsmath,amssymb,amsfonts,amsthm}

% Hyperlinks
\usepackage{hyperref}

% Author details
\title{Analysis of elite runners performances in World Athletics}
\author{02451305}
\date{Compiled: \today}
%%%%%%%%%%%%%%%%%%%%%%%%%%%%%%%%%%%%%%%%%%%%%%%%%%%%%%%%%%%%%%%%%%%%%%

\begin{document}

\maketitle

\textbf{Github Repo:} \href{https://github.com/zakvarty/eds-notes-quarto/releases/tag/v1.2.0-alpha}{REPLACE-WITH-LINK-TO-YOUR-TAGGED-RELEASE}

\section{Project Description (approx. 250 words)}


This project's goal is to give an overview of elite runners performances, by gathering all the information necessary using web 
scraping on the World Athletics website. With this data the aim s identify different profiles of runners and how they can 
improve (or not) and evolve over time.


%=============================================
\pagebreak
%=============================================

\section{Assessment Criteria}


\textbf{Technical Competence:} Proficiency in data collection, processing, analysis, and coding.

\begin{itemize}
    \item Web Scraping from numerous web pages
    \item Preprocessing this scraped data
    \item Analysis of sparse data, dealing with NaNs with statistical models
\end{itemize}

\textbf{User Interface:} Design, functionality, and usability of the final data product.

\begin{itemize}
    \item I used the report form so the functionality and usability of the product lies in the fact that everything is reproducible using the repository
\end{itemize}

\textbf{Analysis and Interpretation:} Depth of analysis, appropriate use of statistical methods, and meaningful interpretation.

\begin{itemize}
    \item Exploratory analysis of the data
    \item Clustering, PCA for the analysis of the perfomances
\end{itemize}

\textbf{Presentation and Communication:} Clarity, organisation and effectiveness of written and visual communication.

\begin{itemize}
    \item Structured report 
\end{itemize}

\textbf{Reproducibility:} Clarity and completeness of documentation for result reproducibility.

\begin{itemize}
    \item Relevant instructions in the README file, ability to run the entire project from (needs a few hours because of the web scraping)
\end{itemize}

\textbf{Version Control:} Effective use of version control systems.

\begin{itemize}
    \item Created branches for each big job: Web scraping, preprocessing, and data generation (NB:the web_scraping branch was deleted after use)
\end{itemize}

%=============================================
\pagebreak
%=============================================

\section{Project Reflection}

\textit{Reflect on the experience of creating your data product. In 6 bullet points and at most 1 page total, summarise the following.} 

\begin{itemize}
    \item \textit{3 things you have learned as part of this process,}
    \item \textit{2 aspects of the project that you found challenging or would approach differently with hindsight,} 
    \item \textit{1 aspect of the project that you would like to learn more about in the future.}
\end{itemize}

\textit{You may delete this italicised text when filling in the template.} 

\textbf{Learnings:}

\begin{itemize}
    \item Item 1
    \item Item 2
    \item Item 3
\end{itemize}

\textbf{Challenges:}

\begin{itemize}
    \item Item 1
    \item Item 2
\end{itemize}

\textbf{Further Development:}

\begin{itemize}
    \item Item 1
\end{itemize}

\end{document}
